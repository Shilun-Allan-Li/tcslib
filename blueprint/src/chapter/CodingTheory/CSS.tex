% chapter/CodingTheory/CSS.tex
% Blueprint for TCSlib.CodingTheory.CSS
%
% This file is meant to be \input{}'d from your main.tex.
% It assumes main.tex uses a class that supports \chapter (e.g. book/report).
% If your main.tex uses article, \chapter will NOT work.

\chapter{CSS codes in the symplectic formalism}

\section{Overview}

This file formalizes the Calderbank--Shor--Steane (CSS) construction inside the symplectic vector space
\[
(\alpha^n)\times(\alpha^n),
\]
equipped with the standard symplectic form. It defines:
\begin{itemize}
  \item the symplectic form \lean{CodingTheory.symplecticForm} and isotropic submodules \lean{CodingTheory.IsIsotropic};
  \item the CSS stabilizer submodule \lean{CodingTheory.cssStabilizer} built from two classical subspaces $C_Z, C_X$;
  \item basic lemmas about the symplectic form on $X$-type and $Z$-type generators
        (\lean{CodingTheory.symplectic_inl_inl}, \lean{CodingTheory.symplectic_inr_inr},
         \lean{CodingTheory.symplectic_inl_inr}, \lean{CodingTheory.symplectic_inr_inl});
  \item the equivalence between the CSS orthogonality condition and isotropy
        (\lean{CodingTheory.css_isotropic_iff_valid});
  \item a dimension formula for the CSS stabilizer (\lean{CodingTheory.finrank_cssStabilizer});
  \item a bridge from matrix commutation to CSS validity (\lean{CodingTheory.matricesCommute_cssValid}, currently unfinished);
  \item (sketch) distance parameters $d_X,d_Z,d_{\mathrm{CSS}}$ and a Steane-code skeleton.
\end{itemize}

Throughout, $\alpha$ is a finite field and $n : \mathbb{N}$.

\section{Core types and symplectic form}

\subsection{Vectors and symplectic vectors}

\begin{definition}[Vectors and symplectic vectors]
\lean{CodingTheory.Vec}
\lean{CodingTheory.SympVec}
A length-$n$ vector over $\alpha$ is \lean{Vec n α} (an alias of \lean{Codeword n α}).
A symplectic vector is a pair \lean{SympVec n α := Vec n α × Vec n α}.
\end{definition}

\subsection{Symplectic form}

\begin{definition}[Symplectic form]
\lean{CodingTheory.symplecticForm}
The symplectic form on \lean{SympVec n α} is
\[
\langle (x,z),(x',z')\rangle
\;=\;
\sum_{i=0}^{n-1} \bigl(x_i z'_i - z_i x'_i\bigr).
\]
\end{definition}

\subsection{Isotropic submodule and stabilizer structure}

\begin{definition}[Isotropic submodule]
\lean{CodingTheory.IsIsotropic}
A submodule $S\le (\alpha^n)\times(\alpha^n)$ is isotropic if the symplectic form vanishes on all pairs of elements of $S$.
\end{definition}

\begin{definition}[Stabilizer structure]
\lean{CodingTheory.Stabilizer}
A stabilizer consists of a submodule $S$ together with a proof that it is isotropic.
\end{definition}

\section{CSS construction from classical subspaces}

\subsection{Dot product}

\begin{definition}[Dot product]
\lean{CodingTheory.dot}
The dot product on $\alpha^n$ is
\[
x\cdot y = \sum_{i=0}^{n-1} x_i y_i.
\]
\end{definition}

\subsection{CSS validity condition}

\begin{definition}[CSS validity condition]
\lean{CodingTheory.cssIsValid}
The CSS condition for classical subspaces $C_Z,C_X \le \alpha^n$ is the orthogonality requirement
\[
\forall x\in C_Z,\ \forall z\in C_X,\quad x\cdot z = 0.
\]
\end{definition}

\subsection{CSS stabilizer submodule}

\begin{definition}[CSS stabilizer submodule]
\lean{CodingTheory.cssStabilizer}
The CSS stabilizer is the submodule of the symplectic space generated by:
\begin{itemize}
  \item $X$-type generators $(x,0)$ for $x\in C_Z$ (image under \lean{LinearMap.inl});
  \item $Z$-type generators $(0,z)$ for $z\in C_X$ (image under \lean{LinearMap.inr}).
\end{itemize}
Formally this is the supremum of the two mapped submodules.
\end{definition}

\section{Symplectic form on generators}

\subsection{Same-type pairings}

\begin{lemma}[$X$--$X$ pairing is zero]
\lean{CodingTheory.symplectic_inl_inl}
\leanok
For all $x,y$, $\langle(x,0),(y,0)\rangle = 0$.
\end{lemma}

\begin{lemma}[$Z$--$Z$ pairing is zero]
\lean{CodingTheory.symplectic_inr_inr}
\leanok
For all $z,w$, $\langle(0,z),(0,w)\rangle = 0$.
\end{lemma}

\subsection{Mixed pairings}

\begin{lemma}[$X$--$Z$ pairing equals dot product]
\lean{CodingTheory.symplectic_inl_inr}
\leanok
For all $x,z$, $\langle(x,0),(0,z)\rangle = x\cdot z$.
\end{lemma}

\begin{lemma}[$Z$--$X$ pairing is minus the dot product]
\lean{CodingTheory.symplectic_inr_inl}
\leanok
For all $z,x$, $\langle(0,z),(x,0)\rangle = - (z\cdot x)$.
\end{lemma}

\section{CSS orthogonality and isotropy}

\subsection{Validity implies isotropy}

\begin{lemma}[Validity implies isotropy]
\lean{CodingTheory.css_isotropic_of_valid}
\leanok
% \uses{\lean{CodingTheory.symplectic_inl_inl},\ \lean{CodingTheory.symplectic_inr_inr},\ \lean{CodingTheory.symplectic_inl_inr},\ \lean{CodingTheory.symplectic_inr_inl}}
If $C_Z \perp C_X$ under the dot product (\lean{cssIsValid}), then the generated CSS stabilizer \lean{cssStabilizer Cz Cx} is isotropic.
\end{lemma}

\subsection{Isotropy implies validity}

\begin{lemma}[Isotropy implies validity]
\lean{CodingTheory.css_valid_of_isotropic}
\leanok
If \lean{cssStabilizer Cz Cx} is isotropic, then $C_Z \perp C_X$ (test isotropy on the generators $(x,0)$ and $(0,z)$).
\end{lemma}

\subsection{Equivalence}

\begin{theorem}[Equivalence]
\lean{CodingTheory.css_isotropic_iff_valid}
\leanok
% \uses{\lean{CodingTheory.css_isotropic_of_valid},\ \lean{CodingTheory.css_valid_of_isotropic}}
For all $C_Z,C_X$,
\[
\mathrm{IsIsotropic}(\mathrm{cssStabilizer}(C_Z,C_X))
\;\Longleftrightarrow\;
\mathrm{cssIsValid}(C_Z,C_X).
\]
\end{theorem}

\section{Dimension of the CSS stabilizer}

\subsection{Trivial intersection of images}

\begin{lemma}[Images intersect trivially]
\lean{CodingTheory.css_images_disjoint}
\leanok
The $X$-type image of $C_Z$ and the $Z$-type image of $C_X$ intersect only at $0$ inside the symplectic space.
\end{lemma}

\subsection{Additivity of finrank}

\begin{theorem}[Finrank of CSS stabilizer is additive]
\lean{CodingTheory.finrank_cssStabilizer}
\leanok
% \uses{\lean{CodingTheory.css_images_disjoint}}
Assuming finite-dimensionality of \lean{Vec n α}, the finrank satisfies
\[
\dim(\mathrm{cssStabilizer}(C_Z,C_X))
=
\dim(C_Z) + \dim(C_X).
\]
\end{theorem}

\section{Matrix presentation bridge}

\subsection{Row space}

\begin{definition}[Row space]
\lean{CodingTheory.rowSpace}
Given a matrix $H$ with $r$ rows and $n$ columns over $\alpha$, its row space is the span of its rows as vectors in $\alpha^n$.
\end{definition}

\subsection{Commutation condition}

\begin{definition}[Matrix commutation]
\lean{CodingTheory.matricesCommute}
For parity-check style matrices $H_X, H_Z$, define
\[
H_X \cdot H_Z^\top = 0.
\]
\end{definition}

\subsection{Commutation implies CSS validity}

\begin{theorem}[Matrix commutation implies CSS validity]
\lean{CodingTheory.matricesCommute_cssValid}
\notready
Intended statement: if $H_X H_Z^\top = 0$, then every row of $H_Z$ is orthogonal (under \lean{dot}) to every row of $H_X$,
and hence \lean{cssIsValid (rowSpace Hz) (rowSpace Hx)} holds by linearity.
\end{theorem}

\section{Weights and distances}

\subsection{Pauli weight}

\begin{definition}[Pauli weight]
\lean{CodingTheory.pauliWeight}
Weight on symplectic vectors counts coordinates where either component is nonzero:
\[
\mathrm{wt}(x,z)=|\{i : x_i\neq 0 \text{ or } z_i\neq 0\}|.
\]
\end{definition}

\subsection{CSS distances}

\begin{definition}[CSS distances]
\lean{CodingTheory.dX}
\lean{CodingTheory.dZ}
\lean{CodingTheory.dCSS}
\notready
Definitions of CSS distances $d_X$ and $d_Z$ via minimization of classical weights over certain candidate sets,
and $d_{\mathrm{CSS}}=\min(d_X,d_Z)$.
\end{definition}

\section{Steane code skeleton over $\mathbb{F}_2$}

\subsection{Steane parity-check matrix}

\begin{definition}[Steane parity-check matrix]
\lean{CodingTheory.H_steane}
A concrete $3\times 7$ matrix over \lean{ZMod 2} representing the Hamming $[7,4,3]$ parity-check matrix (column list form).
\end{definition}

\subsection{Steane commutation}

\begin{theorem}[Steane commutation]
\lean{CodingTheory.steane_commutes}
\leanok
The file includes a proof that \lean{H_steane ⬝ Matrix.transpose H_steane = 0} in \lean{ZMod 2}.
\end{theorem}
