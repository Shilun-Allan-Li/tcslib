\chapter{Coding Theory Basics}
%%%%%%%%%%%%%%%%%%%%%%%%%%%%%%%%%%%%%%%%%%%%%%%%%%%%%%%%%%%%%%%%%%%%%%%%%%%%%%%
\section{Overview}

This file documents the basic definitions and first bounds for $q$-ary block codes over a finite field alphabet.
Throughout, we fix a finite field $\alpha$ (Lean: a type \lean{α} with instances \lean{[Fintype α]} \lean{[Field α]} \lean{[DecidableEq α]} and \lean{[Nonempty α]}).
For a block length $n$, a \emph{codeword} is a function $\mathrm{Fin}\,n \to \alpha$, and a \emph{code} is a finite set of such codewords.

Main results in this file:
\begin{itemize}
  \item Singleton bound (\lean{CodingTheory.singleton_bound}).
  \item Exact size of Hamming balls (\lean{CodingTheory.hamming_ball_size}).
  \item Hamming (sphere-packing) bound (\lean{CodingTheory.hamming_bound}).
  \item Linear-code distance/weight characterization (\lean{CodingTheory.Linear_Code_dist_eq_min_weight}).
  \item Probabilistic distribution lemmas used toward GV-type bounds (\lean{CodingTheory.uniformity_lemma}, \lean{CodingTheory.prob_leq_ball_size}, \lean{CodingTheory.existence_bound}, \lean{CodingTheory.gv_bound}).
  \item List decoding definitions and a (planned) capacity-type existence statement.
\end{itemize}

%%%%%%%%%%%%%%%%%%%%%%%%%%%%%%%%%%%%%%%%%%%%%%%%%%%%%%%%%%%%%%%%%%%%%%%%%%%%%%%
\section{Basic objects: codewords and codes}

\subsection{Codewords}

\begin{definition}[Codeword]
\lean{CodingTheory.Codeword}
A codeword of length $n$ over alphabet $\alpha$ is a function $c : \mathrm{Fin}\,n \to \alpha$.
\end{definition}

\begin{definition}[Pointwise operations]
\lean{CodingTheory.Codeword.add}
\lean{CodingTheory.Codeword.sub}
\lean{CodingTheory.Codeword.zero}
We define pointwise addition/subtraction of codewords and the all-zero codeword.
\end{definition}

\subsection{Codes and linear codes}

\begin{definition}[Code]
\lean{CodingTheory.Codeword.Code}
A (block) code of length $n$ over $\alpha$ is a finite set (\lean{Finset}) of codewords.
\end{definition}

\begin{definition}[Linear code via generator matrix]
\lean{CodingTheory.Codeword.Linear_Code}
A code $C$ is linear with generator matrix $G$ (shape $(n \times m)$) if:
\begin{itemize}
  \item every message $c' : \mathrm{Fin}\,m \to \alpha$ maps to a codeword $G c' \in C$;
  \item every codeword in $C$ is of the form $G c'$ for some message $c'$.
\end{itemize}
\end{definition}

\begin{definition}[Linear code, existential form]
\lean{CodingTheory.Codeword.Linear_Code'}
Existential version of linearity: there exists $m$ and a generator matrix $G$ witnessing linearity as above.
\end{definition}

%%%%%%%%%%%%%%%%%%%%%%%%%%%%%%%%%%%%%%%%%%%%%%%%%%%%%%%%%%%%%%%%%%%%%%%%%%%%%%%
\section{Distance, weight, and rate}

\subsection{Hamming distance and minimum distance predicate}

\begin{definition}[Hamming distance]
\lean{CodingTheory.Codeword.hamming_distance}
Hamming distance between two codewords is the standard \lean{Mathlib.InformationTheory.Hamming.hammingDist}.
\end{definition}

\begin{definition}[Distance predicate for a code]
\lean{CodingTheory.Codeword.distance}
\notready
The predicate \lean{distance C d} asserts:
\begin{itemize}
  \item there exist distinct codewords at distance exactly $d$;
  \item every distinct pair of codewords has distance at least $d$.
\end{itemize}
(So $d$ plays the role of the minimum distance, and the first conjunct ensures it is achieved.)
\end{definition}

\begin{definition}[Weight]
\lean{CodingTheory.Codeword.weight}
Weight of a codeword is its Hamming distance to the all-zero codeword.
\end{definition}

\subsection{Rate and extremal size}

\begin{definition}[Rate]
\lean{CodingTheory.Codeword.rate}
\notready
Rate is defined as
\[
R(C) = \frac{\log |C|}{n \log |\alpha|}.
\]
\end{definition}

\begin{definition}[Maximal size at distance]
\lean{CodingTheory.Codeword.max_size}
\notready
\lean{max\_size n d A} asserts existence of a code of size $A$ and minimum distance $d$ that is maximal among all such codes.
\end{definition}

%%%%%%%%%%%%%%%%%%%%%%%%%%%%%%%%%%%%%%%%%%%%%%%%%%%%%%%%%%%%%%%%%%%%%%%%%%%%%%%
\section{Singleton bound}

\begin{lemma}[Distance is at most block length]
\lean{CodingTheory.dist_le_length}
\leanok
\uses{\lean{CodingTheory.Codeword.distance}}
If a code $C$ has (achieved) minimum distance $d$, then $d \le n$.
\end{lemma}

\begin{theorem}[Singleton bound]
\lean{CodingTheory.singleton_bound}
\leanok
\uses{\lean{CodingTheory.dist_le_length}}
Assuming $\alpha$ is nontrivial, a code $C \subseteq \alpha^n$ with minimum distance $d$ satisfies
\[
|C| \le |\alpha|^{\,n-d+1}.
\]
\end{theorem}

%%%%%%%%%%%%%%%%%%%%%%%%%%%%%%%%%%%%%%%%%%%%%%%%%%%%%%%%%%%%%%%%%%%%%%%%%%%%%%%
\section{Hamming balls and their size}

\subsection{Hamming balls}

\begin{definition}[Hamming ball]
\lean{CodingTheory.hamming_ball}
The Hamming ball of radius $l$ around $c$ is
\[
B_l(c) = \{c' : d(c',c) \le l\},
\]
implemented as a \lean{Finset}.
\end{definition}

\subsection{Exact cardinality}

\begin{theorem}[Cardinality of Hamming balls]
\lean{CodingTheory.hamming_ball_size}
\leanok
For any center $c$,
\[
|B_l(c)|
= \sum_{i=0}^{l} \binom{n}{i} (|\alpha|-1)^i.
\]
\end{theorem}

\subsection{Asymptotic upper bound via q-ary entropy}

\begin{definition}[$q$-ary entropy]
\lean{CodingTheory.Codeword.qaryEntropy}
Define
\[
H_q(p) = p\log_q(q-1) - p\log_q(p) - (1-p)\log_q(1-p).
\]
\end{definition}

\begin{theorem}[Entropy upper bound on Hamming balls]
\lean{CodingTheory.hamming_ball_size_asymptotic_upper_bound}
\notready
Under the usual constraints on $p$ and $q=|\alpha|$, the Hamming ball of radius $\lfloor np\rfloor$ satisfies
\[
|B_{\lfloor np\rfloor}(c)| \le q^{H_q(p)\,n}.
\]
\end{theorem}

%%%%%%%%%%%%%%%%%%%%%%%%%%%%%%%%%%%%%%%%%%%%%%%%%%%%%%%%%%%%%%%%%%%%%%%%%%%%%%%
\section{Entropy algebra lemmas}

\begin{lemma}[Simplifying $q^{H_q(p)}$]
\lean{CodingTheory.q_pow_qary_entropy_simp}
\notready
For $q\ge 2$ and $0<p<1$,
\[
q^{H_q(p)} = (q-1)^p \, p^{-p}\, (1-p)^{-(1-p)}.
\]
\end{lemma}

\begin{lemma}[Same identity, alternate exponentiation form]
\lean{CodingTheory.q_pow_qary_entropy_simp'}
\notready
Variant of the previous lemma using a different exponentiation operator in Lean.
\end{lemma}

%%%%%%%%%%%%%%%%%%%%%%%%%%%%%%%%%%%%%%%%%%%%%%%%%%%%%%%%%%%%%%%%%%%%%%%%%%%%%%%
\section{Analytic helpers (floors, square roots, Stirling)}

\begin{lemma}[Square-root / floor inequality]
\lean{CodingTheory.sqrt_sub_sqrt_floor_le_one}
\notready
For $x\ge 0$,
\[
\sqrt{x} - \sqrt{\lfloor x\rfloor} \le 1.
\]
\end{lemma}

\begin{lemma}[Asymptotic lower bound on binomial term]
\lean{CodingTheory.binomial_coef_asymptotic_lower_bound'}
\notready
A Stirling-based statement giving (eventual) lower bounds of the form
\[
\binom{n}{\lfloor np\rfloor}(q-1)^{pn}
\;\gtrsim\;
q^{H_q(p)n - \varepsilon(n)}
\quad\text{with }\varepsilon(n)=o(n).
\]
\end{lemma}

%%%%%%%%%%%%%%%%%%%%%%%%%%%%%%%%%%%%%%%%%%%%%%%%%%%%%%%%%%%%%%%%%%%%%%%%%%%%%%%
\section{Sphere packing: disjointness of balls and Hamming bound}

\begin{lemma}[Different codewords have disjoint decoding balls]
\lean{CodingTheory.hamming_ball_non_intersect}
\leanok
If $C$ has minimum distance $d>0$, then for distinct $c_1,c_2\in C$,
\[
B_{\left\lfloor \frac{d-1}{2}\right\rfloor}(c_1)
\cap
B_{\left\lfloor \frac{d-1}{2}\right\rfloor}(c_2)
=\varnothing.
\]
\end{lemma}

\begin{lemma}[Disjointness packaged as \lean{Disjoint}]
\lean{CodingTheory.hamming_ball'_disjoint}
\leanok
A \lean{Finset.Disjoint} reformulation of the previous lemma.
\end{lemma}

\begin{theorem}[Hamming (sphere-packing) bound]
\lean{CodingTheory.hamming_bound}
\leanok
\uses{\lean{CodingTheory.hamming_ball_size}, \lean{CodingTheory.hamming_ball'_disjoint}}
Let $q=|\alpha|>1$. If $C\subseteq\alpha^n$ has minimum distance $d>0$, then
\[
|C|
\le
\frac{q^n}{\sum_{i=0}^{\left\lfloor\frac{d-1}{2}\right\rfloor}\binom{n}{i}(q-1)^i}.
\]
\end{theorem}

%%%%%%%%%%%%%%%%%%%%%%%%%%%%%%%%%%%%%%%%%%%%%%%%%%%%%%%%%%%%%%%%%%%%%%%%%%%%%%%
\section{Linear codes: distance equals minimum nonzero weight}

\begin{theorem}[Distance equals minimum nonzero weight in linear codes]
\lean{CodingTheory.Linear_Code_dist_eq_min_weight}
\leanok
\uses{\lean{CodingTheory.Codeword.Linear_Code'}, \lean{CodingTheory.Codeword.weight}, \lean{CodingTheory.Codeword.distance}}
If $C$ is linear and has (achieved) minimum distance $d$, then:
\begin{itemize}
  \item every nonzero codeword in $C$ has weight at least $d$;
  \item there exists a codeword in $C$ of weight exactly $d$.
\end{itemize}
\end{theorem}

%%%%%%%%%%%%%%%%%%%%%%%%%%%%%%%%%%%%%%%%%%%%%%%%%%%%%%%%%%%%%%%%%%%%%%%%%%%%%%%
\section{Random generator matrices and uniformity}

\subsection{Uniform distributions (as functions)}

\begin{definition}[Uniform distribution on vectors]
\lean{CodingTheory.uniform_vector_dist}
\notready
Uniform distribution on $\alpha^n$ encoded as a function $\alpha^n \to \mathbb{R}$.
\end{definition}

\begin{lemma}[Finiteness of constrained matrices]
\lean{CodingTheory.finite_matrix_dist}
\leanok
For fixed $x$ and $v$, the set of matrices $G$ with $Gx=v$ is finite.
\end{lemma}

\begin{definition}[Distribution induced by random matrices]
\lean{CodingTheory.matrix_dist}
\notready
Define $\mu_x(v)$ as the fraction of $n\times k$ matrices $G$ satisfying $Gx=v$.
\end{definition}

\begin{definition}[Row extraction]
\lean{CodingTheory.get_matrix_row}
Utility function returning the $i$-th row as a $1\times k$ matrix.
\end{definition}

\begin{theorem}[Uniformity of $Gx$ for nonzero $x$]
\lean{CodingTheory.uniformity_lemma}
\notready
If $x\neq 0$ and $k\ge 1$, then for uniform random $G$, the product $Gx$ is uniformly distributed over $\alpha^n$.
\end{theorem}

%%%%%%%%%%%%%%%%%%%%%%%%%%%%%%%%%%%%%%%%%%%%%%%%%%%%%%%%%%%%%%%%%%%%%%%%%%%%%%%
\section{Toward Gilbert--Varshamov: probability and counting lemmas}

\begin{theorem}[Bounding probability of small weight]
\lean{CodingTheory.prob_leq_ball_size}
\notready
A bound comparing the fraction of matrices $G$ with $\mathrm{wt}(Gx)<d$
to the relative size of a Hamming ball in $\alpha^n$.
\end{theorem}

\begin{theorem}[Existence bound via union bound]
\lean{CodingTheory.existence_bound}
\notready
Counts matrices that map \emph{some} nonzero message to a codeword of weight $<d$.
\end{theorem}

\begin{theorem}[Gilbert--Varshamov-style counting lower bound]
\lean{CodingTheory.gv_bound}
\notready
Under a condition relating $k$ to $n$ and the Hamming ball volume,
there exists a generator matrix $G$ whose associated linear code has distance at least $d$.
\end{theorem}

%%%%%%%%%%%%%%%%%%%%%%%%%%%%%%%%%%%%%%%%%%%%%%%%%%%%%%%%%%%%%%%%%%%%%%%%%%%%%%%
\section{List decoding}

\begin{definition}[List-decodable codes]
\lean{CodingTheory.list_decodable}
A code $C$ is $(\rho,L)$-list-decodable if every Hamming ball of radius $\lfloor\rho n\rfloor$
contains at most $L$ codewords from $C$.
\end{definition}

\begin{lemma}[Positivity of $q$-ary entropy expression]
\lean{CodingTheory.qary_entropy_pos}
\leanok
A positivity lemma for the entropy expression under the standard parameter range.
\end{lemma}

\begin{theorem}[List decoding capacity (existence)]
\lean{CodingTheory.list_decoding_capacity}
\leanok
A existence theorem producing codes of rate approximately
$1 - H_q(p) - 1/L$ that are $p$-list-decodable with list size $L$.
\end{theorem}
