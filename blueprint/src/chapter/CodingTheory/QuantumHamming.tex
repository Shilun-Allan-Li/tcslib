% chapter/CodingTheory/QuantumHamming.tex
% Blueprint for TCSlib.CodingTheory.QuantumHamming
%
% Intended to be \input{}'d from your main.tex (book/report class recommended).

\chapter{Quantum Hamming bound for stabilizer codes}

\section{Overview}

This chapter formalizes a $q$-ary version of the quantum Hamming bound for stabilizer codes
in the symplectic formalism developed in \lean{TCSlib.CodingTheory.CSS}.
The main ingredients are:
\begin{itemize}
  \item the logical dimension parameter $k$ computed from the stabilizer rank;
  \item Pauli error balls in the symplectic space and their cardinalities;
  \item a notion of quantum distance defined by minimizing Pauli weight among
        operators commuting with the stabilizer but not contained in it;
  \item disjointness of correctable error balls (under suitable distance hypotheses);
  \item the final counting argument giving the Hamming bound (for a non-degenerate/pure code).
\end{itemize}

Throughout, $\alpha$ is a finite field with $q=\lvert \alpha\rvert$, and we work over the symplectic space
\[
(\alpha^n)\times(\alpha^n),
\]
which is the Lean type \lean{SympVec n α}.

\section{Imports and dependencies}

This file imports \lean{TCSlib.CodingTheory.Basic} (classical codewords/weights)
and \lean{TCSlib.CodingTheory.CSS} (symplectic form, Pauli weight on \lean{SympVec}, stabilizers).

\section{Quantum code dimension}

\begin{definition}[Quantum code dimension]
\lean{CodingTheory.quantum_code_dimension}
Assuming \lean{FiniteDimensional α (SympVec n α)}, define
\[
k := n - \dim_\alpha(S),
\]
where $S$ is the stabilizer submodule (a submodule of the symplectic space) and $\dim_\alpha(S)$ is
\lean{FiniteDimensional.finrank α S}.
\end{definition}

\begin{remark}
This matches the usual $[[n,k,d]]$ stabilizer convention when the stabilizer has rank $r=\dim S$.
\end{remark}

\section{Pauli error balls}

\subsection{Definition}

\begin{definition}[Pauli error ball]
\lean{CodingTheory.pauli_error_ball}
For $t\in\mathbb{N}$ and center $p$ in the symplectic space, the error ball is the finite set
\[
B_t(p)=\{p' : \mathrm{wt}(p'-p)\le t\},
\]
implemented as a \lean{Finset} via \lean{Set.toFinset}.
\end{definition}

\subsection{Size formula}

\begin{definition}[Closed form for the size]
\lean{CodingTheory.pauli_error_ball_size}
The intended size for a $q$-ary symplectic alphabet is
\[
|B_t|=\sum_{i=0}^{t} \binom{n}{i}(q^2-1)^i.
\]
\end{definition}

\subsection{Translation invariance}

\begin{lemma}[Error-ball size is independent of the center]
\lean{CodingTheory.pauli_error_ball_size_independent}
\leanok
For all $p,q$, one has $\lvert B_t(p)\rvert=\lvert B_t(q)\rvert$, via translation
$p' \mapsto p' + (q-p)$.
\end{lemma}

\subsection{Cardinality agrees with the closed form}

\begin{lemma}[Cardinality of the ball]
\lean{CodingTheory.pauli_error_ball_card}
\notready
The file proves (or aims to prove) that for each $t$ and $p$,
\[
\lvert B_t(p)\rvert = \text{\lean{pauli_error_ball_size}\,$t$}
=\sum_{i=0}^{t} \binom{n}{i}(q^2-1)^i.
\]
This is a counting argument by choosing the support set of size $i$ and
assigning a nonzero element of $\alpha\times\alpha$ at each chosen coordinate.
\end{lemma}

\section{Quantum distance}

\subsection{Definition}

\begin{definition}[Quantum distance]
\lean{CodingTheory.quantum_distance}
Given a stabilizer \lean{S : Stabilizer n α}, define the candidate set of Pauli operators
$p$ satisfying:
\begin{itemize}
  \item $p$ commutes with every stabilizer element (symplectic form equals $0$);
  \item $p \notin S$ (not a stabilizer element);
  \item $\mathrm{wt}(p)\neq 0$ (nontrivial).
\end{itemize}
Then \lean{quantum_distance S} is the minimum Pauli weight among candidates,
or $0$ if there are no candidates.
\end{definition}

\subsection{Upper bounds from witnesses}

\begin{lemma}[Distance is at most the weight of any valid witness]
\lean{CodingTheory.quantum_distance_le_pauliWeight}
\leanok
If $p$ commutes with the stabilizer, $p\notin S$, and $\mathrm{wt}(p)\neq 0$, then
\[
\text{\lean{quantum_distance S}} \le \text{\lean{pauliWeight}\,p}.
\]
\end{lemma}

\subsection{Commutation is stable under subtraction}

\begin{lemma}[Symplectic form is additive in the left argument under subtraction]
\lean{CodingTheory.symplecticForm_sub_left}
\leanok
\[
\langle u-v, w\rangle = \langle u,w\rangle - \langle v,w\rangle.
\]
\end{lemma}

\begin{lemma}[If two operators commute with the stabilizer, so does their difference]
\lean{CodingTheory.commutesWithStabilizer_sub}
\leanok
If $p$ and $q$ commute with every $s\in S$, then $p-q$ also commutes with every $s\in S$.
\end{lemma}

\subsection{Distance bound for differences of commuting operators}

\begin{lemma}[Distance is bounded by the weight of a difference]
\lean{CodingTheory.quantum_distance_le_weight_sub}
\leanok
If $p,q$ commute with the stabilizer, and $p-q\notin S$ with nonzero weight, then
\[
\text{\lean{quantum_distance S}} \le \mathrm{wt}(p-q).
\]
\end{lemma}

\begin{lemma}[Convenient inequality form]
\lean{CodingTheory.pauliWeight_sub_ge_of_commuting}
\leanok
If $d \le \text{\lean{quantum_distance S}}$, then under the same hypotheses,
\[
d \le \mathrm{wt}(p-q).
\]
\end{lemma}

\section{Disjointness of correctable error balls}

\begin{lemma}[Disjointness of error balls]
\lean{CodingTheory.pauli_error_ball_disjoint}
\notready
Given a distance parameter $d>0$ and two distinct stabilizer elements $p_1\neq p_2$,
if $\mathrm{wt}(p_1-p_2)\ge d$, then the balls of radius
\[
t=\left\lfloor \frac{d-1}{2}\right\rfloor
\]
around $p_1$ and $p_2$ are disjoint:
\[
B_t(p_1)\cap B_t(p_2)=\varnothing.
\]
The proof follows the standard triangle-inequality argument: any common element would
force $\mathrm{wt}(p_1-p_2)\le \mathrm{wt}(x-p_1)+\mathrm{wt}(x-p_2)\le 2t < d$.
\end{lemma}

\section{Auxiliary inequality for Pauli weight}

\begin{lemma}[Pauli-weight triangle inequality]
\lean{CodingTheory.pauliWeight_triangle}
\notready
The development includes a lemma of the form
\[
\mathrm{wt}(p-q) \le \mathrm{wt}(p)+\mathrm{wt}(q),
\]
used to bound weights of differences in subsequent disjointness / packing arguments.
\end{lemma}

\section{Quantum Hamming bound}

\subsection{Statement}

\begin{theorem}[Quantum Hamming bound for stabilizer codes]
\lean{CodingTheory.quantum_hamming_bound}
\notready
Let $S$ be a stabilizer code on $n$ qudits over $\alpha$ with $q=\lvert\alpha\rvert>1$.
Assume:
\begin{itemize}
  \item $k = \text{\lean{quantum_code_dimension}\,S.S}$;
  \item a distance parameter $d>0$ with $d\le n$ and $d\le \text{\lean{quantum_distance S}}$;
  \item \emph{purity / non-degeneracy}:
    for any $e\in S.S$ with $\mathrm{wt}(e)<d$, one has $e=0$.
\end{itemize}
Let $t=\left\lfloor \frac{d-1}{2}\right\rfloor$. Then
\[
q^k \cdot \sum_{i=0}^{t} \binom{n}{i}(q^2-1)^i \;\le\; q^n.
\]
\end{theorem}

\subsection{Proof structure}

The proof follows the standard sphere-packing / syndrome-counting argument:
\begin{enumerate}
  \item Compute the ball size using \lean{pauli_error_ball_card} and rewrite it as the closed form sum.
  \item Define the \emph{syndrome map} sending a Pauli operator $p$ to the linear functional
        $s \mapsto \langle p, s\rangle$ on the stabilizer $S.S$.
        This is implemented as a linear map into \lean{S.S →ₗ[α] α}.
  \item Bound the size of the syndrome space using finite-dimensional linear algebra, yielding
        $\lvert S.S \to \alpha\rvert = q^{n-k}$ (the code constraints).
  \item Prove injectivity of the syndrome map on the radius-$t$ ball centered at $0$,
        using:
        \begin{itemize}
          \item if two errors have the same syndrome, their difference commutes with the stabilizer;
          \item if the difference is not in the stabilizer, its weight is at least the distance;
          \item but the triangle/ball radius bounds force its weight $<d$;
          \item purity then forces the difference to be $0$.
        \end{itemize}
  \item Conclude $\lvert B_t\rvert \le q^{n-k}$, then multiply by $q^k$ to obtain the final inequality.
\end{enumerate}
