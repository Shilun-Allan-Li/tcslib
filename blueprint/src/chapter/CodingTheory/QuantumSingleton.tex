% chapter/CodingTheory/QuantumSingleton.tex
% Blueprint for TCSlib.CodingTheory.QuantumSingleton
%
% Intended to be \input{}'d from your main.tex (book/report class recommended).

\chapter{Quantum Singleton bound via symplectic methods}

\section{Overview}

This chapter develops a proof skeleton of the quantum Singleton bound for stabilizer codes:
\[
k \le n - 2(d-1),
\]
for a $[[n,k,d]]$ stabilizer code (over a finite field $\alpha$ with $q = |\alpha|$),
using the symplectic formalism from \lean{TCSlib.CodingTheory.CSS} and the quantum-distance
machinery from \lean{TCSlib.CodingTheory.QuantumHamming}.

The Lean file focuses on:
\begin{itemize}
  \item bilinearity and nondegeneracy of the symplectic form;
  \item defining the symplectic dual $S^{\perp_\omega}$ of a submodule $S$;
  \item an ``erasure correctability from distance'' lemma;
  \item a support-restriction submodule with an explicit dimension computation;
  \item a rank bound extracted from commutator relations (a symplectic-pairing argument);
  \item reducing Singleton to a stabilizer-rank lower bound of the form
        $2(d-1)\le \mathrm{rank}(S)$, then translating to the usual inequality on $k$.
\end{itemize}

\section{Imports and dependencies}

This file imports \lean{TCSlib.CodingTheory.QuantumHamming}, hence depends on:
\lean{TCSlib.CodingTheory.CSS} (symplectic form, Pauli weight, stabilizers) and
\lean{TCSlib.CodingTheory.Basic}.

\section{Symplectic form bilinearity}

\subsection{A bundled bilinear form}

\begin{definition}[Bundled symplectic bilinear form]
\lean{TCSlib.CodingTheory.QuantumSingleton.symplecticBilin}
The symplectic form \lean{symplecticForm} is packaged as a bilinear form
\lean{LinearMap.BilinForm α (SympVec n α)} with the four linearity fields proven by
expanding sums over coordinates and using ring normalization.
\end{definition}

\subsection{Convenience lemmas}

The file records standard consequences of bilinearity:

\begin{lemma}[Additivity in the left argument]
\lean{TCSlib.CodingTheory.QuantumSingleton.symplecticForm_add_left}
\leanok
\[
\langle x+y, z\rangle = \langle x,z\rangle + \langle y,z\rangle.
\]
\end{lemma}

\begin{lemma}[Additivity in the right argument]
\lean{TCSlib.CodingTheory.QuantumSingleton.symplecticForm_add_right}
\leanok
\[
\langle x, y+z\rangle = \langle x,y\rangle + \langle x,z\rangle.
\]
\end{lemma}

\begin{lemma}[Homogeneity in the left argument]
\lean{TCSlib.CodingTheory.QuantumSingleton.symplecticForm_smul_left}
\leanok
\[
\langle c\cdot x, y\rangle = c\,\langle x,y\rangle.
\]
\end{lemma}

\begin{lemma}[Homogeneity in the right argument]
\lean{TCSlib.CodingTheory.QuantumSingleton.symplecticForm_smul_right}
\leanok
\[
\langle x, c\cdot y\rangle = c\,\langle x,y\rangle.
\]
\end{lemma}

There are also finite-sum versions:

\begin{lemma}[Sum in the left argument]
\lean{TCSlib.CodingTheory.QuantumSingleton.symplecticForm_sum_left}
\leanok
\[
\left\langle \sum_{i\in s} f(i),\, y\right\rangle
= \sum_{i\in s}\langle f(i),y\rangle.
\]
\end{lemma}

\begin{lemma}[Sum in the right argument]
\lean{TCSlib.CodingTheory.QuantumSingleton.symplecticForm_sum_right}
\leanok
\[
\left\langle x,\, \sum_{i\in s} f(i)\right\rangle
= \sum_{i\in s}\langle x,f(i)\rangle.
\]
\end{lemma}

\begin{lemma}[Weighted sum in the left argument]
\lean{TCSlib.CodingTheory.QuantumSingleton.symplecticForm_smul_sum_left}
\leanok
\[
\left\langle \sum_{i\in s} \ell(i)\,f(i),\, y\right\rangle
= \sum_{i\in s}\ell(i)\,\langle f(i),y\rangle.
\]
\end{lemma}

\section{Symplectic dual}

\begin{definition}[Symplectic dual]
\lean{TCSlib.CodingTheory.QuantumSingleton.symplecticDual}
For a submodule $S \le SympVec n \alpha$, define
\[
S^{\perp_\omega} := \{v \mid \forall s\in S,\ \langle v,s\rangle = 0\},
\]
as a \lean{Submodule} with closure under $0$, addition, and scalar multiplication
proved using bilinearity.
\end{definition}

\begin{lemma}[Stabilizer isotropy]
\lean{TCSlib.CodingTheory.QuantumSingleton.stabilizer_is_isotropic}
\leanok
Restates \lean{S.isIso} in a convenient $\forall s\in S, \forall t\in S$ form.
\end{lemma}

\begin{lemma}[Stabilizer is contained in its symplectic dual]
\lean{TCSlib.CodingTheory.QuantumSingleton.stabilizer_le_symplecticDual}
\leanok
If $S$ is a stabilizer (isotropic), then $S \le S^{\perp_\omega}$.
\end{lemma}

\section{Erasure correction from distance}

\subsection{Correctability predicate}

\begin{definition}[Erasure correctability]
\lean{TCSlib.CodingTheory.QuantumSingleton.ErasureCorrectable}
Given an erased coordinate set $E\subseteq \{1,\dots,n\}$, we say $E$ is correctable for
a stabilizer $S$ if every Pauli operator $p$ supported inside $E$ that commutes with the
stabilizer must already lie in the stabilizer.
\end{definition}

\subsection{Distance implies erasure correctability}

\begin{lemma}[Erasure correctability from quantum distance]
\lean{TCSlib.CodingTheory.QuantumSingleton.erasure_correctable_of_dist}
\leanok
Let $d$ denote the quantum distance of $S$, written in Lean as
\lean{quantum_distance}$\,(S)$.
If $|E| < d$, then $E$ is erasure-correctable.

The proof uses:
\begin{itemize}
  \item support $\Rightarrow$ weight bound $\mathrm{wt}(p)\le |E|$;
  \item if $p\notin S$ and $p$ commutes with $S$, then $d \le \mathrm{wt}(p)$
        (via \lean{quantum_distance_le_pauliWeight});
  \item contradiction with $|E|<d$.
\end{itemize}
\end{lemma}

\subsection{Restriction operators and cleaning}

The file introduces:
\begin{itemize}
  \item \lean{restrictTo}: zeroes out coordinates outside the erased set;
  \item \lean{symplecticFormOn}: symplectic form summed only over coordinates in a set;
  \item lemmas relating restriction to the restricted symplectic sum, and a
        ``cleaning'' lemma producing a stabilizer element that cancels a logical operator on $E$.
\end{itemize}

\begin{lemma}[Restriction preserves the restricted form]
\lean{TCSlib.CodingTheory.QuantumSingleton.symplecticFormOn_restrict}
\leanok
\[
\langle \mathrm{restrict}(p), q\rangle_{E} = \langle p,q\rangle_{E}.
\]
\end{lemma}

\begin{lemma}[Restricted logical operator is a stabilizer]
\lean{TCSlib.CodingTheory.QuantumSingleton.restrict_logical_to_erased_in_stabilizer}
\notready
Assuming erasure correctability on $E$ and a commuting logical operator $L$,
the restriction $\mathrm{restrictTo}(L,E)$ lies in the stabilizer.
\end{lemma}

\begin{lemma}[Cleaning lemma]
\lean{TCSlib.CodingTheory.QuantumSingleton.clean_logical_operator}
\notready
There exists $s\in S$ such that $L+s$ vanishes on the erased set $E$ coordinatewise.
\end{lemma}

\section{Support submodule and its dimension}

\subsection{Submodule of vectors supported on a set}

\begin{definition}[Support submodule]
\lean{TCSlib.CodingTheory.QuantumSingleton.supportSubmodule}
For $C\subseteq \{1,\dots,n\}$, define the submodule consisting of symplectic vectors
supported inside $C$:
\[
V_C := \{v \mid \forall j\notin C,\ v(j)=0\}.
\]
\end{definition}

\subsection{Isomorphism to functions on C}

The file constructs:
\begin{itemize}
  \item a restriction linear map \lean{restrictToC};
  \item an extension linear map \lean{extendFromC};
  \item inverse lemmas \lean{restrictToC_extendFromC} and \lean{extendFromC_restrictToC};
  \item the resulting linear equivalence \lean{supportSubmoduleEquiv}.
\end{itemize}

\begin{lemma}[Dimension of the support submodule]
\lean{TCSlib.CodingTheory.QuantumSingleton.finrank_supportSubmodule}
\leanok
\[
\dim_\alpha(V_C) = 2\,|C|.
\]
\end{lemma}

\section{Decomposing the symplectic form and nondegeneracy}

\begin{lemma}[Decomposition over a partition]
\lean{TCSlib.CodingTheory.QuantumSingleton.symplecticForm_decompose}
\notready
If $A,B,C$ are pairwise disjoint and $A\cup B\cup C = \mathrm{univ}$, then
\[
\langle p,q\rangle = \langle p,q\rangle_A + \langle p,q\rangle_B + \langle p,q\rangle_C.
\]
\end{lemma}

\begin{lemma}[Nondegeneracy]
\lean{TCSlib.CodingTheory.QuantumSingleton.symplecticForm_nondegenerate}
\leanok
If $\forall v,\ \langle u,v\rangle=0$, then $u=0$.
The proof uses special test vectors supported on a single coordinate to read off
each component of $u$.
\end{lemma}

\section{Dimension of the symplectic dual and logical space}

\begin{lemma}[Dimension of the symplectic dual]
\lean{TCSlib.CodingTheory.QuantumSingleton.finrank_symplecticDual}
\notready
For finite-dimensional ambient space,
\[
\dim(S^{\perp_\omega}) = 2n - \dim(S).
\]
\end{lemma}

\begin{lemma}[Logical space dimension identity]
\lean{TCSlib.CodingTheory.QuantumSingleton.finrank_logical_space}
\notready
For a stabilizer $S$,
\[
\dim(S^{\perp_\omega}) - \dim(S) = 2\,(n-\dim(S)) = 2k,
\]
where $k$ denotes the quantum code dimension of $S.S$,
written in Lean as \lean{quantum_code_dimension}\,(S.S).
\end{lemma}

\section{Symplectic bases}

\begin{definition}[Symplectic basis in a submodule]
\lean{TCSlib.CodingTheory.QuantumSingleton.SymplecticBasis}
A structure packaging $k$ symplectic pairs $(X_i,Z_i)$ inside a submodule $V$
such that:
\[
\langle X_i, Z_j\rangle = \delta_{ij},\quad
\langle X_i,X_j\rangle=0,\quad
\langle Z_i,Z_j\rangle=0.
\]
\end{definition}

\section{A rank bound from commutator relations}

\begin{lemma}[Rank bound from commutators (support version)]
\lean{TCSlib.CodingTheory.QuantumSingleton.rank_bound_from_commutators_simple}
\notready
Given $k$ symplectic pairs $(X_i,Z_i)$ supported inside a coordinate set $C$ and satisfying
the canonical commutation relations, one proves:
\[
k \le |C|.
\]
\end{lemma}

\section{From stabilizer rank to the Singleton bound}

\subsection{Singleton from a stabilizer rank lower bound}

\begin{lemma}[Singleton bound from rank inequality]
\lean{TCSlib.CodingTheory.QuantumSingleton.quantum_singleton_bound_of_rank}
\leanok
Assuming
\[
2(d-1) \le \dim(S),
\]
and $k=n-\dim(S)$, deduce
\[
k \le n - 2(d-1).
\]
\end{lemma}

\subsection{Distance implies stabilizer rank lower bound}

\begin{lemma}[Rank lower bound from distance]
\lean{TCSlib.CodingTheory.QuantumSingleton.stabilizer_rank_ge_of_distance}
\notready
Let $d$ denote the quantum distance of $S$, written in Lean as
\lean{quantum_distance}\,(S).
Assuming a hypothesis \lean{h_basis} producing symplectic pairs with controlled support,
one shows
\[
2(d-1) \le \dim(S).
\]
\end{lemma}

\subsection{Main theorem}

\begin{theorem}[Quantum Singleton bound]
\lean{TCSlib.CodingTheory.QuantumSingleton.quantum_singleton_bound}
\notready
Let $d$ denote the quantum distance of $S$, written in Lean as
\lean{quantum_distance}\,(S).
Under the standard parameter assumptions ($0<d\le n$)
and the auxiliary basis-existence hypothesis \lean{h_basis}, the file concludes:
\[
k \le n - 2(d-1).
\]
\end{theorem}
